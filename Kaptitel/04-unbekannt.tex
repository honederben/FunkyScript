


% 07-01-2020

\begin{Bsp}{(Ortsoperator in der Quantenmechanik)\\}
    $\mathcal{H} = L^2(\mathbb{R})$ und $\hat{Q} = (D,Q)$ definiert durch

    $(Qf)(x) = x f(x)$

    $D(\hat{Q} = C_0^\infty (\mathbb{R})= \{f | f \textrm{ ist } C^\infty$ mit kompaktem Träger\}

    $\hat{Q}$ ist symmetrisch (aber nicht selbstadj.)
\end{Bsp}

\begin{Bsp}
    $\mathcal{H} = L^2(\mathbb{R})$ , $\hat{A} = (D,A)$

    $(Af)(x) = x^{-\alpha} f(x) \quad \quad \alpha > 0, \textrm{ fest}$

    $D = \{f \in L^2(\mathbb{R}) | f=0 \textrm{ in Umgebung von } x = 0 \}$

    $\Rightarrow D^\ast = \{g \in L^2(\mathbb{R}) | x^{- \alpha} \cdot g \in L^2(\mathbb{R})\} \supset D$

    und $\hat{A} = (D^\ast, A^\dagger) = (D^\ast, A) \underbrace{\supset}_{Erweiterung} \hat{A} \quad \Rightarrow \hat{A} \textrm{ ist symm.}$ 

    $$\langle g , Af \rangle = \int_{-\infty}^\infty \,dx \, g^\ast(x) (x^{- \alpha} f(x)) = 
    \int_{-\infty}^\infty \,dx \, (x^{- \alpha} g)^\ast(x) f(x)) = \int_{-\infty}^\infty \,dx (A^\dagger g)(x) f(x)$$
\end{Bsp}

\begin{Def}
    Ein linearer Operator $\hat{A} = (D,A)$ in $\mathcal{H}$ heißt
    \begin{enumerate}[label=(\alph*)]
        \item abgeschlossen, geschrieben als $\overline{\hat{A}} = \hat{A}$, genau dann, wenn für jede Folge
                $\{x_n\}_{n \in \mathbb{N}} \subset D$, für die $\lim_{n \rightarrow \infty} {\Vert x_n -x \Vert}_\mathcal{H} = 0$ und $\lim_{n \rightarrow \infty} {\Vert Ax_n -y \Vert}_\mathcal{H} = 0$
                (mit $x,y \in^{n \rightarrow \infty} \mathcal{H}$) gilt, dass $x \in D$ und $Ax = y$
        \item abschließbar geanu dann, wenn $\hat{A}$ eine abgeschlossene Erweiterung 
                $\hat{B} = \overline{\hat{B}} \supset \hat{A}$ besitzt
    \end{enumerate}
\end{Def}

\begin{Bem}
    Wesentlich (nur bei Abgeschlossenheit gefordert)

    $A( \lim_{n \rightarrow \infty} x_n) = \lim_{n \rightarrow \infty} (A x_n)$

    nur für jene Cauchyfolgen $\{ x_n\}_{n \in \mathbb{N}}$ gilt, für die auch 
    $\{ A x_n\}_{n \in \mathbb{N}}$ in $\mathcal{H}$ konvergiert.
\end{Bem}

\begin{Bsp}
    $L^2[0;1]$

    $\hat{A}_j = (D_j,A) \quad\quad j=1, \,2 \quad\quad (Af)(x) = x^{-\alpha} f(x) \quad \alpha >0 $

    $D_1 = \{ f \in L^2[0;1] \,|\, f = 0 \textrm{ in Umgebung von } x=0 \}$
    
    $D_2 = \{ f \in L^2[0;1] \,|\, x^{- \alpha} f \in L^2[0;1] \}$

    $D_1 \subset D_2 \quad \textrm{und} \quad \overline{D_1} = \overline{D_2} = L^2[0;1]$

    \textbf{$\hat{A_1}$ ist nicht abgeschlossen:} $f(x) = x^\alpha \notin D_1$
    \begin{addmargin}[25pt]{0pt}
        $f_n(x) = \left\{ \begin{array}{ll}
            1 & \frac{1}{n} < x \leq 1  \\
            0 & \, \textrm{sonst} \\
            \end{array}
            \right. \quad \in D_1$

        $(Af_n)(x) = \left\{ \begin{array}{ll}
            1 & \frac{1}{n} < x \leq 1  \\
            0 & \, \textrm{sonst} \\
            \end{array}
            \right. \quad \in L^2[0;1]$         
            
        $\lim_{n \rightarrow \infty} f_n = f \quad \in \mathcal{H}$\\
        $\textrm{  } \hspace{5em} \,\, f \quad \notin D_1 \Rightarrow \hat{A} \textrm{ nicht abgeschlossen}$

        $\lim_{n \rightarrow \infty} Af_n = 1 \quad \in \mathcal{H}$
    \end{addmargin}


    \textbf{$\hat{A_2}$ ist abgeschlossen:}
    \begin{addmargin}[25pt]{0pt}
        $\{g_n\}_{n \in \mathbb{N}} \subset D_2$ mit
        $$ \lim_{n \rightarrow \infty} g_n = g \in L^2[0;1] \quad \textrm{und}$$
        $$ \lim_{n \rightarrow \infty} Ag_n = G \in L^2[0;1]$$
        
        Z.z.: $g \in D_2$ und $Ag=G$ Es gilt $D_2 = D_1^\ast = D(\hat{A_1}^ \dagger)$

        $\forall h \in D_1: \quad \langle Ah, g \rangle = \langle x^{-\alpha} h, g \rangle =
        \lim_{n \rightarrow \infty} \langle x^{-\alpha} h, g_n \rangle =
        \lim_{n \rightarrow \infty} \langle h, x^{-\alpha} g_n \rangle$\\ 
        $\textrm{ } \qquad \qquad \quad = \lim_{n \rightarrow \infty} \langle h, A g_n \rangle =
        \langle h , G \rangle$

        $\Rightarrow y \in D_1^\ast = D_2 \quad \textrm{und} \quad G = Ag$
    \end{addmargin}
\end{Bsp}

\begin{Satz}
    $\hat{A} = (D,A)$ dicht definiert in $\mathcal{H}$. Dann gilt:
    \begin{enumerate}[label=(\alph*)]
        \item $\hat{A} \subseteq \hat{B} \Rightarrow \hat{B}^\dagger \subseteq \hat{A}^\dagger$
        \item der zu $\hat{A}$ adjuingierte Operator  $\hat{A}^\dagger$ ist abgeschlossen:
                $\overline{\hat{A}^\dagger} = \hat{A}^\dagger$
        \item $\hat{A}$ ist genau dann abgeschlossen, wenn $D^\ast = D(\hat{A}^\dagger)$ dicht in
                $\mathcal{H}$ ist. In diesem Fall ist $\overline{\hat{A}} =
                \left(\hat{A}^\dagger\right)^\dagger$
    \end{enumerate}
\end{Satz}

\begin{Bem}{$\textrm{ }$ \newline} %WA Zeilenumbruch
    \begin{itemize}
        \item (b) folgt aus Stetigkeit des Skalarprodukts
        \item Ein selbstadjungierter Operator ist immer abgeschlossen, da $\hat{A} = \hat{A}^\dagger$ und
                in (b)  $A^\dagger$ abgeschlossen.
        \item Ein dichtdefinierter Operator in $\mathcal{H}$ heißt wesentlich selbstadjungiert, wenn 
                $\hat{A}$ symmetrisch und $\overline{\hat{A}}$ selbstadjungiert.
    \end{itemize}
\end{Bem}

\subsubsection{ÜberblicK über Begriffe symmetrisch abgeschlossen und (wesentlich) selbstadjungiert}
\begin{itemize}
    \item $\hat{A}$ symm: $\underbrace{\hat{A} \subset \overline{\hat{A}}}_{\hat{A} \textrm{ hat 
            abgeschl. Erweiterung}} = \left(\hat{A}^\dagger\right)^\dagger \subset \hat{A}^\dagger$
    \item $\hat{A}$ symm. \& abgeschl.: $\hat{A} = \overline{\hat{A}} = \left(\hat{A}^\dagger\right)^\dagger
            \subset \hat{A}^\dagger$
    \item $\hat{A}$ wesentl. selbstadj: $\hat{A} \subset \overline{\hat{A}} =
            \left(\hat{A}^\dagger\right)^\dagger = \hat{A}^\dagger$
    \item $\hat{A}$ selbstadj.: $\hat{A} = \overline{\hat{A}} = \left(\hat{A}^\dagger\right)^\dagger =
            \hat{A}^\dagger$
\end{itemize}

Hauptproblem für selbstadjungierten Operator: $D = D(\hat{A}) = D^\ast = D(\hat{A}^\dagger)$

\begin{Satz}
    Für einen symmetrischen Operar $\hat{A}$ in $\mathcal{H}$ sind folgende Aussagen äquivalent:
    \begin{enumerate}[label=(\alph*)]
        \item $\hat{A}$ selbstadjungiert: $\hat{A} = \hat{A}^\dagger$
        \item $\hat{A}$ abgeschlossen $\left( \overline{\hat{A}} = \hat{A} \right)$ und
                $Ker(\hat{A}^\dagger \pm iI) = \{0\}$
        \item $Ran(\hat{A}^\dagger \pm iI) = \mathcal{H}$
    \end{enumerate}
\end{Satz}

\begin{Bsp}{quantenmechanisches Teilchen in 1 Raumdimension, das im Intervall $[0;1]$ "eingesperrt" ist\\}

\end{Bsp}