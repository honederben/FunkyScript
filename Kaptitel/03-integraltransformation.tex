\section{Integral Transformationen}
    $f \in L^1[-\frac{L}{2}, \frac{L}{2}]$ lässt sich durch
    FR darstellen. Was passiert mit FR, wenn 
    $L \rightarrow \infty$ ?


    %ToDo
    \subsection{Testfunktionenräume und Distributionen}

    \begin{Def}
        V VR über $\mathbb{K}$. Ein Funktional $F$ ist eine Abbildung:
        $$F: V \rightarrow \mathbb{K}$$
        $$ \phi \in V \mapsto F(\phi) = \alpha \in \mathbb{K}$$

        Ein lineares Funktional $F$ erfüllt die Linearitätsbedingung
        $$F(\alpha \phi + \beta \psi) = \alpha F(\phi) + \beta F(\psi) \quad \phi , \psi \in V \quad \alpha, \beta \in \mathbb{K} $$

        Die Menge aller linerarer Funktionale auf $V$
        $$V^\ast =  \{ F | F:V \rightarrow \mathbb{K} \quad linear \} $$

        bildet selbst bzgl. der "punktweisen" Addition und Multiplikation
        mit einem Skalar einen VR, den algebraischen Dualraum von $V$.
        $$(F+G)(\phi) = F(\phi) + G(\phi) \quad \forall \phi \in V $$
        $$(\alpha F)(\phi) = \alpha (F)(\phi)  \quad \forall \alpha \in \mathbb{K} \quad \forall \phi \in V $$
    \end{Def}

    \begin{Def}
        Der Raum der beliebig oft stetig diff-baren (komplex- oder 
        reellwertigen) Funktionen mit kompakten Trägern ist definiert als
        $$C_0^\infty (\mathbb{R}^n)  := \{ f:\mathbb{R}^n \rightarrow \mathbb{C}(\mathbb{R}) | f \in C^\infty(\mathbb{R}^n) \textrm{ und } f(\vec{x})=0 \textrm{ für } \vec{x} \in \mathbb{R}^n \setminus  \Omega \textrm{ mit } \Omega \textrm{ kompakt} \}$$
    \end{Def}