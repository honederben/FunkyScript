\section{Hilberträume}
    \begin{Def}
		Ein metrischer Raum besteht aus eine Menge $X$ und einer Abb. $d: X,X\rightarrow \mathbb{R}$
		die jedem geordneten Paar von Elementen aus $X$ eine reele Zahl Zuordnet.

		Diese Abb soll ($\forall x,y,z \in X$) folgende Eigenschaften bessitzen:

		\begin{itemize}
			\item $d(x,y) \geq 0$ (Nichtnegativität)
			\item $d(x,y) = 0 \Leftrightarrow x = y$ (Eindeutigkeit)
			\item $d(x,y) = d(y,x)$ (Symetrie)
			\item $d(x,y) + d(y,z) \geq d(x,z)$ (Dreiecksungleichung)
		\end{itemize}
	\end{Def}
	
	\begin{Def}
		Ein normierter Raum ist ein Vektorraum $V$ über den Köprer $\mathbb{C} (\mathbb{R})$
		auf dem eine Abb. $\Vert\cdot\Vert: V \rightarrow \mathbb{R}$ erklärt ist, die jedem
		Element $x \in V$ eine reele Zahl $\Vert x \Vert$ zuordnet und folgende Eigenschaften
		besitzt
		\begin{itemize}
			\item $\Vert x \Vert \geq 0$ (Nichtnegativität)
			\item $\Vert x \Vert = 0 \Leftrightarrow x = 0$ (Eindeutigkeit)
			\item $\Vert \alpha \cdot x \Vert = \vert \alpha \vert \cdot \Vert x \Vert$ (Skalierung)
			\item $\Vert x + y\Vert \leq \Vert x \Vert + \Vert y \Vert$ (Dreiecksungleichung)
		\end{itemize}
	\end{Def}

	\begin{Bem}
		Jeder normierte Raum ist ein metrischer Raum mit Metrik $d(x,y) = \Vert x-y \Vert$
		durch die Norm induzierte Metrik. 
	\end{Bem}

	\begin{Def}
		Ein normierter Raum, in dem jede Cauchy-Folge konvergiert (bzgl. Metrik $d(x,y) = \Vert x-y \Vert$)
		ist ein vollständig normierter Raum bzw. Banachraum.
	\end{Def}

	% \begin{Bsp}
	% 	\begin{itemize}
	% 		\item $\mathbb{R}^n: \Vert \vec x \Vert = \sqrt{\sum_{i=1}^n x_i^2}$
	% 		\item $\mathbb{C}^n: \Vert \vec z \Vert = \sqrt{\sum_{i=1}^n z_i^2}$
	% 		\item Hilbert'scher Folgenraum $\ell^2$
	% 	\end{itemize}
	% \end{Bsp}

	\textbf{Beispiele}
	\begin{itemize}
		\item $\mathbb{R}^n: \Vert \vec x \Vert = \sqrt{\sum_{i=1}^n x_i^2}$
		\item $\mathbb{C}^n: \Vert \vec z \Vert = \sqrt{\sum_{i=1}^n z_i^2}$
		\item Hilbert'scher Folgenraum $\ell^2$
	\end{itemize}

	\begin{Def}
		Sei $V$ ein Vektorraum über $\mathbb{C} (\mathbb{R})$.
		Ein Skalarprodukt auf $V$ ist eine Abb. die jedem geordneten Paar von
		Elementen aus $V$ eindeutig ein, mit $\langle x,y\rangle$ bezeichnetes Element aus
		$\mathbb{C} (\mathbb{R})$ zuordnet und folgende Eigenschaften erfüllt.

		$\langle\cdot,\cdot \rangle: V\cdot V \rightarrow \mathbb{C}$ 
		
		$\forall x,y,z \in V \quad \forall \lambda \in \mathbb{C} (\mathbb{R})$

		\begin{itemize}
			\item $\langle x,y\rangle = \langle y,x\rangle^\ast$
			\item $\langle x,\lambda y\rangle = \lambda \langle x,y\rangle$
			\item $\langle x,y+z\rangle = \langle x,y\rangle + \langle x,z\rangle$
			\item $\langle x,x\rangle \geq$ für $ x \in \mathbb{R}$
			\item $\langle x,x\rangle = 0 \Leftrightarrow x=0$
		\end{itemize}
	\end{Def}

	\begin{Bem}
		Skalarprodukt ist 'positiv hermitische Form'
		\begin{itemize}
			\item $\langle \lambda x,y\rangle = \langle y,\lambda x\rangle^\ast = \lambda^\ast \langle y, x\rangle^\ast=
					\lambda^\ast \langle x,y\rangle$
			\item $\langle x+y,z\rangle = \langle z,x+y\rangle^\ast + \langle x,z\rangle$
		\end{itemize}
	\end{Bem}

